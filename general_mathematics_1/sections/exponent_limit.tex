\section{حد های نمایی}

برای حل حدود نمایی از فرمول های $\ln$ باید استفاده بکنیم.


\begin{tikzpicture}
	\begin{axis}[gstyle]
		\addplot [name path=lb, color=blue, domain=0.001:5]{ln(x)} node{};
		\addplot [color=red, domain=-5:5]{e^x} node{};
	\end{axis}
\end{tikzpicture}

تابع $ln$ همان طور که در نمودار مشخص است معکوس تابع نمایی $e^x$ است. این تابع در $x > 0$ تعریف می‌شود.

مقادیر معروف $ln$:
\begin{align*}
&\ln 0 = -\infty \\
&\ln 1 = 0 \\
&\ln e = 1 \\
&\ln \infty = \infty \\
\end{align*}

با استفاده از تابه $ln$ می‌توانیم توابع نمایی را به ضرب تبدیل کنیم. داریم $\ln f^g = g\ln f$ و $e^{\ln u} = u$.
با استفاده از این دو فرمول برخی از حدود مبهم رفع ابهام می‌شوند. مثال می‌زنیم:

مثال ۱:
\begin{align*}
&\lim_{x\to 0} (1 + \sqrt{x})^{\frac{\sin\sqrt{x}}{x}} = 1^{\infty} \to \\
&\lim_{x\to 0} \ln (1 + \sqrt{x})^{\frac{\sin\sqrt{x}}{x}} = \ln A \to \\
&\lim_{x\to 0} {\frac{\sin\sqrt{x}}{x}} \times \ln(1 + \sqrt{x}) = \ln A \to \\
&\lim_{x\to 0} {\frac{\sin\sqrt{x}}{x}} \times \lim_{x\to 0} \ln(1 + \sqrt{x}) = \ln A \to \\
&\lim_{x\to 0} {\frac{\sin\sqrt{x}}{x}} = 1 \quad \lim_{x\to 0} \ln(1 + \sqrt{x}) = 0 \to \\
&\lim_{x\to 0} 1 \times 0 = \ln A \\
&ln A = 0 \to A = e^0 = 1
\end{align*}

زمانی که به $x$ مقدار صفر را می‌دهیم، جواب حد مبهم می‌شود و نیاز به رفع ابهام خواهد داشت.
جواب حد را $A$ در نظر می‌گیریم و از عبارت $ln$ میگیریم. حال هر دو طرف $ln$ دارند و می‌توانیم از فرمول $\ln f^g = g\ln f$ استفاده کنیم و توان را به ضرب تبدیل کنیم.
بعد از تبدیل به ضرب، حد را حل کرده و بر اساس فرمول $e^{\ln u} = u$ جواب نهایی را بدست می‌آوریم.

در مثال بالا از قاعده هوپیتال استفاده نکردیم. اما درحالتی که باز هم جواب حد مبهم باشد، با همین الگوریتم از هوپیتال استفاده خواهیم کرد.

مثال ۱:


\begin{align*}
\lim_{x\to\infty} \frac{1+x}{2+x}^{\frac{1-\sqrt{x}}{1-x}}
\end{align*}