\newpage
\subsection{برگه تقلب فرمول های انتگرال}
\begin{align*}
&\int x^n dx = \frac{1}{n+1}x^{n+1}+c \\
&\int x^{-1} dx = \int \frac{1}{x} dx = \ln|x|+c \\
&\int \sin (ax+b) dx = \frac{-1}{a} \cos(ax+b)+c \to \int \sin x dx = -\cos x+c \\
&\int \cos (ax+b) dx = \frac{1}{a} \sin(ax+b)+c  \to \int \cos x dx = \sin x+c \\
&\int \tan x dx = \int \frac{\sin x}{\cos x} dx = -\ln|\cos x|+c \\
&\int \cot x dx = \int \frac{\cos x}{\sin x} dx = \ln|\cos x|+c \\
&\int \sec^2 dx = \int (1+\tan^2 x) dx = \int \frac{1}{\cos^2 x} dx = \tan x + c \\
&\int \csc^2 dx = \int (1+\cot^2 x) dx = \int \frac{1}{\sin^2 x} dx = -\cot x + c \\
&\int \sec x.\tan x dx = \sec x + c \\
&\int \csc x.\cot x dx = \csc x + c \\
&\int \sec x dx = \ln|\sec x+\tan x|+c \\
&\int \csc x dx = \ln|\csc x-\cot x|+c \\
&\int \sinh (ax+b) dx = \frac{1}{a} \cosh (ax+b) + c \to \int \sinh x dx = \cosh x + c \\
&\int \cosh (ax+b) dx = \frac{1}{a} \sinh (ax+b) + c \to \int \cosh x dx = \sinh x + c \\
&\int \frac{dx}{\sqrt{a^2-x^2}} = \sin^{-1} (\frac{x}{a}) + c = -\cos^{-1} (\frac{x}{a}) + c \\
&\int \frac{dx}{x\sqrt{x^2-a^2}} = \frac{1}{a} \sec^{-1} (\frac{x}{a}) + c \\
&\int \frac{dx}{x^2+a^2} dx = \frac{1}{a} \tan^{-1} (\frac{x}{a}) + c \\
&\int \frac{dx}{\sqrt{x^2+a^2}} = \sinh^{-1} (\frac{x}{a}) + c \\
&\int \frac{dx}{\sqrt{x^2-a^2}} = \cosh^{-1} (\frac{x}{a}) + c \\
&\int u'e^u dx = e^u + c \to \int e^{ax+b} dx = \frac{1}{a} e^{ax+b} + c \to \int e^x dx = e^x + c \\
&\int a^u dx = \frac{1}{\ln|a|} + c \to \int a^x \ln|a| dx = \frac{1}{\ln|a|} a^x + c  \\
&\int \ln|x| dx = \ln|x| dx = x\ln|x| - x + c \\
\end{align*}