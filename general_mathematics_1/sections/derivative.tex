\chapter{Derivative}
\section{Definition}
Derivative limit definition:
\begin{align*}
&\frac{df}{dx} = f'(x) \\
&f'(x_0) = \lim_{x\to x_0} \frac{f(x) - f(x_0)}{x-x_0} \\
&f'(x) = \lim_{h\to 0} \frac{f(x+h) - f(x)}{h} \\
\end{align*}

Proof of the Derivative of Constant: $\frac{df}{dx}(c)$
\begin{align*}
f(x) &= c \\
f'(x) &= \lim_{h\to 0}\frac{f(x+h) - f(x)}{h} \\
&= \lim_{h\to 0}\frac{c - c}{h} \\
&= \lim_{h\to 0} 0 = 0
\end{align*}

\section{Common formulas}
\subsection{1}
\[ (y = c)\to y'= 0 \]

Example:
\[ (\sqrt{2})' = (\frac{3}{2})' = (\sin^{-1}\frac{1}{10})' = 0 \]

Note: $(\sin\infty)'$ does not exists.\footnote{\href{https://math.stackexchange.com/questions/635135/infinite-derivatives-of-a-trigonometric-function}{https://math.stackexchange.com/questions/635135/infinite-derivatives-of-a-trigonometric-function}}

\subsection{2}
\[ (ax^n)' = a(x^n)' = anx^{n-1} \]

Note: The $n$ coefficient have no effect on derivative.

Examples:
\begin{align*}
&(3x^7)' = 21x^6 \\
&(2x^{-9})' = -18x^{-10} \\
&(x^{2}\sqrt[7]{x^3})'=
	(x^2 x^{\frac{3}{7}})' =
	(x^{\frac{17}{7}})' =
	\frac{17}{7} x^{\frac{10}{7}} \\
\end{align*}

\subsection{3}
\[ (u \pm v)' = u' \pm v' \]

Example:
\[ (z + 1)' = (z)' + (1)' = 1 + 0 = 1 \]

\subsection{4}
\[ (uv)' = u'v + uv' \]

Example:
\begin{align*}
(\underset{u}{x}\underset{v}{\sin x})' = u'v + uv' = 
\left\lbrace
\begin{array}{r@{}l}
	u' &= 1 \\
	v' &= \cos x
\end{array}
\right.
\end{align*}

\subsection{5}
\[ (au^n)' = a(u^n)' = anu^{n-1}u' \]

Example:
\[ (3(x^2-x^{-2}+\sqrt{3})^{10})' = anu^{n-1}u' = 3(u^9u') \to u' = 2x +2x^{-3} + 0 \]

Example for derivation with radical:
\begin{align*}
(3\sqrt[7]{(x^{-3}+\frac{1}{x})^2})' &= anu^{n-1}u' \\
&= 3(\underbrace{(x^{-3}+\frac{1}{x})^2}_u)^{\frac{2}{7}} \to \\
u' &= x^{-3}\times\frac{1}{x} =
\underbrace{x^{-3}}_{w_1} \times \underbrace{x^{-1}}_{w_2} \\
&= w_1'w_2+w_1w_2' = 
\left\lbrace
\begin{array}{r@{}l}
	w_1' &= -3x^{-4} \\
	w_2' &= -x^{-2}
\end{array}
\right.
\end{align*}
