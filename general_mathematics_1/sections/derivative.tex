\chapter{Derivative}
\section{Definition}
Derivative limit definition:
\begin{align*}
\frac{df}{dx} &= f'(x) \\
f'(x_0) &= \lim_{x\to x_0} \frac{f(x) - f(x_0)}{x-x_0} \\
f'(x) &= \lim_{h\to 0} \frac{f(x+h) - f(x)}{h} \\
\end{align*}

Proof of the Derivative of Constant: $\frac{df}{dx}(c)$
\begin{align*}
f(x)&=c \\
f'(x) &= \lim_{h\to 0}\frac{f(x+h) - f(x)}{h} \\
&= \lim_{h\to 0}\frac{c - c}{h} \\
&= \lim_{h\to 0} 0 = 0
\end{align*}

\section{Common formulas}
\label{1}
\[ (y = c)\to y'= 0 \]

Example:
\[ (\sqrt{2})' = (\frac{3}{2})' = (\sin^{-1}\frac{1}{10})' = 0 \]

Note: $(\sin\infty)'$ does not exists.\footnote{\href{https://math.stackexchange.com/questions/635135/infinite-derivatives-of-a-trigonometric-function}{https://math.stackexchange.com/questions/635135/infinite-derivatives-of-a-trigonometric-function}}

