\section{مشتق}
\subsection{تعریف}
تعریف حدی مشتق:
\begin{align*}
&\frac{df}{dx} = f'(x) \\
&f'(x_0) = \lim_{x\to x_0} \frac{f(x) - f(x_0)}{x-x_0} \\
&f'(x) = \lim_{h\to 0} \frac{f(x+h) - f(x)}{h} \\
\end{align*}

اثبات فرمول مشتق عدد ثابت: $\frac{df}{dx}(c)$
\begin{align*}
f(x) &= c \\
f'(x) &= \lim_{h\to 0}\frac{f(x+h) - f(x)}{h} \\
&= \lim_{h\to 0}\frac{c - c}{h} \\
&= \lim_{h\to 0} 0 = 0
\end{align*}

\subsection{فرمول های متعارف}
\subsubsection{$y'$}
\[ (y = c)\to y'= 0 \]

مثال:
\[ (\sqrt{2})' = (\frac{3}{2})' = (\sin^{-1}\frac{1}{10})' = 0 \]

Note: $(\sin\infty)'$ does not exists.\footnote{\href{https://math.stackexchange.com/questions/635135/infinite-derivatives-of-a-trigonometric-function}{https://math.stackexchange.com/questions/635135/infinite-derivatives-of-a-trigonometric-function}}

\subsubsection{$(ax^n)'$}
\[ (ax^n)' = a(x^n)' = anx^{n-1} \]

Note: The $n$ coefficient have no effect on derivative.

مثال:s:
\begin{align*}
&(3x^7)' = 21x^6 \\
&(2x^{-9})' = -18x^{-10} \\
&(x^{2}\sqrt[7]{x^3})'=
	(x^2 x^{\frac{3}{7}})' =
	(x^{\frac{17}{7}})' =
	\frac{17}{7} x^{\frac{10}{7}} \\
\end{align*}

\subsubsection{$(u \pm v)'$}
\[ (u \pm v)' = u' \pm v' \]

مثال:
\[ (z + 1)' = (z)' + (1)' = 1 + 0 = 1 \]

\subsubsection{$(uv)'$}
\[ (uv)' = u'v + uv' \]

مثال:
\begin{align*}
(\underset{u}{x}\underset{v}{\sin x})' = u'v + uv' = 
\left\lbrace
\begin{array}{r@{}l}
	u' &= 1 \\
	v' &= \cos x
\end{array}
\right.
\end{align*}

\subsubsection{$(au^n)'$}
\[ (au^n)' = a(u^n)' = anu^{n-1}u' \]

مثال:
\[ (3(x^2-x^{-2}+\sqrt{3})^{10})' = anu^{n-1}u' = 3(u^9u') \to u' = 2x +2x^{-3} + 0 \]

مثال: for derivation with radical:
\begin{align*}
(3\sqrt[7]{(x^{-3}+\frac{1}{x})^2})' &= anu^{n-1}u' \\
&= 3(\underbrace{(x^{-3}+\frac{1}{x})^2}_u)^{\frac{2}{7}} \to \\
u' &= x^{-3}\times\frac{1}{x} =
\underbrace{x^{-3}}_{w_1} \times \underbrace{x^{-1}}_{w_2} \\
&= w_1'w_2+w_1w_2' = 
\left\lbrace
\begin{array}{r@{}l}
	w_1' &= -3x^{-4} \\
	w_2' &= -x^{-2}
\end{array}
\right.
\end{align*}

\subsubsection{$(\frac{u}{v})'$}
\[ (\frac{u}{v})' = \frac{u'v - uv'}{v^2} \to (\frac{1}{x})' = \frac{-1}{x^2} \]

مثال:
\begin{align*}
(\frac{\overbrace{1-2x^{-7}}^u}{\underbrace{2x^3-\frac{2}{\sqrt{3}}}_v})' = \frac{u'v - uv'}{v^2} \to
\left\lbrace
\begin{array}{r@{}l}
	u' &= 0 + 14x^{-8} = 14x^{-8} \\
	v' &= 6x^2-0 = 6x^2
\end{array}
\right.
\end{align*}

\subsubsection{$(\sqrt{u})'$}
\[ (\sqrt{u})' = \frac{u'}{2\sqrt{u}} \to (\sqrt{x})' = \frac{1}{2\sqrt{x}} \]

مثال:
\begin{align*}
&(\sqrt{\underbrace{x^2-x+x^{-4}}_u})' = \frac{u'}{2\sqrt{u}} \\
&u' = 2z-1-4x^{-5}
\end{align*}

\subsubsection{$(\ln u)'$}
\[ (\ln u)' = \frac{u'}{u} \to (\ln x)' = \frac{1}{x} \]

\subsubsection{$(a^u)'$}
\[ (a^u)' \overset{a>0}{\longrightarrow} u'a^u\ln a \to (a^x)' = a^x\ln a \]

\subsubsection{$(e^u)'$}
\[ (e^u)' = u'e^u\ln e = u'e^u \to (e^{ax+b})' = ae^{ax+b} \]

\subsubsection{$(u^v)'$}
\[ (u^v)' \overset{u>0}{\longrightarrow} u^v(v'\ln u+\frac{u'v}{u}) \to (x^x)' = (1 + \ln x) \]

% Explain Ln and log

\subsection{مشتق توابع مثلثاتی}
\subsubsection{$(\sin u)'$}
\[ (\sin u)' = u'\cos u \longrightarrow (sin(ax+b))' = a \cos(ax+b) \longrightarrow (\sin x)' = \cos x \]

مثال:
\[ (\sin(\ln x))' = (\sin u)' = u'cosu \to u' = \frac{1}{x}\]

\subsubsection{$(\cos u)'$}
\begin{align*}
&(\cos u)' = -u'\sin u \longrightarrow (cos(ax+b))' = -a\sin(ax+b) \to (\cos x)' = -\sin x \\
&(\cos^n x) = (\cos x)^n \\
\end{align*}

مثال:
\[ ((\cos x)^3)' = (u^3)' = 3u^2u' \to u' = -\sin x \]

\subsubsection{$(\tan u)'$}
\[ (\tan u)' = u'(1+\tan^2u) = u'\sec^2u \to (\tan x)' = 1+\tan^2 x = \sec^2 x \]

\subsubsection{$(\cot u)'$}
\[ (\cot u)' = -u'(1+\cot^2u) = -u'\csc^2u \to (\cot x)' = -(1+\cot^2x) = -\csc^2x \]

\subsubsection{$(\sec u)'$}
\[ (\sec u)' = u'.\sec u.\tan u \to (sec x)' = x'\sec x.\tan x \]

مثال:
\[ (\sec(\csc x))' = (\sec u)' = u'\sec u\tan u \to u' = -\cot x \csc x \]


\subsubsection{$(csc u)'$}
\[ (csc u)' = -u'.\csc u.\cot u \to (\csc x)' = -\cot x.\csc x \]

% Explain trigonometrics

\subsection{مشتق توابع معکوس مثلثاتی}
\subsubsection{$(\sin^{-1} u)'$}
\[ (\sin^{-1} u)' = \frac{u'}{\sqrt{1-u^2}} \to (\sin^{-1}x)' = \frac{1}{\sqrt{1-x^2}} \]

\subsubsection{$(\cos^{-1} u)'$}
\[ (\cos^{-1} u)' = \frac{-u'}{\sqrt{1-u^2}} \to (\cos^{-1}x)' = \frac{-1}{\sqrt{1-x^2}} \]

\subsubsection{$(\tan^{-1} u)'$}
\[ (\tan^{-1} u)' = \frac{u'}{1+u^2} \to (\tan^{-1}x)' = \frac{1}{1+x^2} \]

\subsubsection{$(\cot^{-1} u)'$}
\[ (\cot^{-1} u)' = \frac{-u'}{1+u^2} \to (\cot^{-1}x)' = \frac{-1}{1+x^2} \]

\subsubsection{$(\sec^{-1}u)'$}
\[ (\sec^{-1}u)' = \frac{u'}{u\sqrt{u^2-1}} \to (\sec^{-1}x)' = \frac{1}{x\sqrt{x^2-1}} \]

\subsubsection{$(\csc^{-1}u)'$}
\[ (\csc^{-1}u)' = \frac{-u'}{u\sqrt{u^2-1}} \to (\csc^{-1}x)' = \frac{-1}{x\sqrt{x^2-1}} \]

% Inverse trigonometric functions

\subsection{مشتق توابع هایپربولیک}
\subsubsection{$(\sinh u)'$}
\[ (\sinh u)' = u'\cosh u \to (\sinh(ax+b))' = a\cosh(ax+b) \to (\sinh x)' = \cosh x \]

\subsubsection{$(\cosh u)'$}
\[ (\cosh u)' = u'\sinh u \to (\cosh(ax+b))' = a\sinh(ax+b) \to (\cosh x)' = \sinh x \]

% ...
