\subsection{تحلیل و تعریف خواسته ها}
مجموعه ای از آدم ها و دسته بندی ها نیاز های مختلفی دارند که CRM باید آنها را برطرف کند. برای مثال سازمان نیاز های خاص خود را دارد، محصول و مدیریت آن نیاز های خود را دارد برای پیاده سازی
و در نهایت تمام اینها برای برطرف سازی نیاز کاربران است.

\subsubsection{خواسته های سازمان}
سازمان یک سری نیازمندی دارد که بر اساس آنها سیستم CRM را توسعه می‌دهد.
سیستم نیاز دارد تا بتواند تمام دپارتمان های خود را به درستی مدیریت کند همانطور که در مقدمه مستند اشاره کردیم.

\bt{لیست نیاز های سیستم:}
\begin{itemize}
	\item ثبت و مدیریت خودکار ورود و خروج کارمندان
	\item محاسبه و پرداخت حقوق کارمندان به صورت خودکار و دوره ای
	\item تهیه گزارشات مختلف از نیرو ها و عملکرد سیستم
	\item معرفی بهترین نیرو ها از نظر تاثیر گذاری در سازمان
	\item معرفی نیرو های ضعیف تر برای برنامه ریزی در جهت پیشرفت آنها
	\item مشاهده نمودار های پیشرفت شرکت
	\item مشاهده نمودار های پیشرفت نیرو ها از نظر کاری
	\item دنبال کردن هزینه های انجام شده در سیستم
	\item دنبال کردن سود های دریافتی در سیستم
	\item سرشماری کارمندان و جستجو در اطلاعات آنها
	\item بایگانی اطلاعات حسابرسی و مدیریتی
	\item سود مالی و افزایش بودجه شرکت بواسطه فروش CRM
	\item عدم خرید و هزینه غیر ضروری به سازمان های دیگر و برون سپاری
\end{itemize}

\subsubsection{خواسته های محصول}

برای توسعه محصول، نیاز به فراهم سازی اجزا مختلفی وجود دارد و باید آنها را فراهم کرد.
نیاز های محصول اغلب فنی هستند و لیستی که از آن تهیه می‌شود لیستی فنی از نیازمندی های شرکت برای تولید محصول است.

\bt{نیاز های محصول}
\begin{itemize}
	\item تشکلی تیم Architect برای دیزاین کردن یک سیستم درست نرم افزاری
	\item تشکیل تیم UI/UX و دیزاین های لازم قبل از توسعه
	\item تشکیل تیم توسعه نرم افزار بخش back end
	\item تشکیل تیم توسعه نرم افزار بخش Front end
	\item تشکیل تیم Operation برای اماده سازی شبکه و سرور ها
	\item تشکیل تیم DevOps برای خودکار سازی فراند ها
	\item تشکیل تیم QA و تست برای برسی جنبه های مختلف نرم افزار
	\item تشکیل تیمی برای مانیتورینگ و برسی مداوم عملکرد محصول
\end{itemize}

\subsubsection{خواسته های کاربر}

نیاز های کاربر مطابق آن چیزی هست که سازمان گفته است. البته که سیستم CRM ما انعطاف پذیری بالایی دارد و معماری خود را Pluggable پیش می‌برد.
به این صورت که بخشی برای پلاگین ها دارد و می‌توان عملکرد های مختلف را به آن اضافه کرد.
همین مورد باعث می‌شود خواسته های مختلف کاربر برطرف شود.
این خواسته ها باید طی فرایند تحقیق بعد از اولین رلیز برسی شود و فعلا نمیتواند داخل مستند قرار داشته باشد.
