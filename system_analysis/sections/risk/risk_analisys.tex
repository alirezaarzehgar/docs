\subsection{تحلیل و دسته بندی ریسک ها}
مشکلات و ریسک های احتمالی را در نسخه MVP سیستم شناسایی و دسته بندی کردیم. حال نوبع تحلیل آنها و برسی این است که هر ریسک چقدر احتمال وقوع دارد
و چقدر می‌تواند آسیب بزند. این مورد بسیار حائز اهمیت است.

\subsubsection{آسیب پذیری سخت افزار}
به مراتب احتمال وقوع مشکلات سخت افزاری نسبت به ریسک های دیگر کمتر است. اما اگر آن را نسبت به زمانی که سیستم مورد استفاده قرار می‌گیرد
و خساراتی که به سیستم می‌زند بسنجیم، متوجه می‌شویم بسیار مهم است. درصد کمی احتمال دارد یک مشکل سخت افزاری در این سیستم رخ بدهد. اما فرضا در طول ۱۰ سال
استفاده یک مشتری از این سیستم، یک درصد احتمال هم به معنی بار ها اتفاق افتادن مشکل هست.

\risktbl{خرابی سرور پروداکشن}{>1\%}{غیر قابل تحمل}

مثال های زیادی وجود دارد که در آن سرور پروداکشن دچار مشکل می‌شود. برای مثال ممکن است به هر دلیلی یکی از اجزا اصلی سرور دچار مشکل شود.
مثلا RAM یا CPU سرور خراب شود. در این حالت تیم Operation باید چاره اندیشی کند و به فکر این مشکل باشد که زمان و هزینه زیادی مصرف خواهد شد.

\risktbl{رفتن برق دیتاسنتر}{>10\%}{حداقل بودن ریسک}

رفتن برق دیتاسنتر ها گاها اتفاق میوفتد و دور از انتظار نخواهد بود. تیم فنی و Operation باید به دنبال راهکاری باشند که چنین مشکلی پیش نیاید.

\risktbl{خرابی حافظه سرور دیتابیس}{>1\%}{غیر قابل تحمل}

به هر دلیلی ممکن است Storage سرور پروداکشن یا دیتابیس دچار مشکل شده و تمام داده ها از بین برود. تیم فنی باید راهکاری پیدا کند که در صورت وقوع چنین حادثه ای،‌
دیتا کاربران از بین نرود.

\risktbl{اختلال در شبکه دیتاسنتر}{>1\%}{غیر قابل تحمل}

در مواقعی شبکه داخلی یک سازمان یا دیتاسنتر دچار مشکل می‌شود. در این حالت مدیران فنی به سرعت باید مشکل را کشف و حل کنند.

\subsubsection{آسیب پذیری نرم افزار}
بر خلاف مشکلات سخت افزاری، مشکلات نرم افزاری بسیار احتمال وقوع بیشتری دارند و حتی می‌توانند هزینه بیشتری را به دوش شرکت بیندازند. 

\risktbl{وجود انواع باگ های عملکردی در سیستم و ایجاد اختلال}{>20\%}{حداقل بودن ریسک}

باگ ها و مشکلات نرم افزاری همیشه بودند. اما در صورتی که به سادگی بتوان از آنها سو استفاده کرد و تعداد آنها بسیار زیاد باشد، سیستم دچار مشکل بزرگی می‌شود.
تمام تلاش تیم توسعه نرم افزار باید این باشد که محصولی توسعه بدهد که کمترین باگ ممکن را داشته باشد.

\risktbl{حملات DoS}{>10\%}{غیر قابل تحمل}

در صورتی که سیستم دچار حمله DoS بشود، کاربران دیگر نمی‌توانند برای مدتی از سرویس استفاده بکنند. این مشکلی به شدت جدیست که باید از سمت
تیم Operation و Security چاره اندیشی بشود.

\risktbl{پر شدن ترافیک سرور}{0\%}{قابل قبول}

یک سیستم داخلی CRM بعید است بخاطر ترافیک زیاد داون شود. از این رو. به همین دلیل نیازی به نگرانی از این بابت نخواهد بود. در صورتی که سیستم در گستره بزرگ تری
مورد استفاده قرار گیرد نیز نیاز به تدابیر بیشتری خواهد داشت.

\risktbl{ایجاد مشکل در فرایند دیپلوی کردن نرم افزار}{<30\%}{حداقل بودن ریسک}

زمانی که نسخه جدید از نرم افزار توسعه پیدا می‌کند، مشکلاتی برای دیپلوی کردن آن نسخه اتفاق خواهد افتاد. برای مثال ممکن است  نسخه دیپلوی شده باگ داشته باشد
یا حتی تیم Operation نتواند به درستی آن را Deploy کند. در این حالت باید از تیم های DevOps کمک گرفت. در این حالت احتمال مشکل در فرایند Deployment
کمتر و کمتر خواهد شد.

\risktbl{مورد حمله قرار گرفتن سیستم}{>20\%}{غیر قابل تحمل}

همواره کسانی وجود دارند که با نیت های مختلف علاقمند هستند به سازمان ها نفوذ کنند. سیستم CRM ما نیز از این خطر دور نیست. من باب رفع خطر، باید از تیم فنی و امنیت 
کمک گرفته شود و تدابیری صورت بگیرد.

\subsubsection{آسیب پذیری اپراتور}
در صورتی که سیستم به درستی طراحی نشود، اپراتور ها سهوا یا عمدا می‌توانند به سیستم آسیب بزنند. در جدول فوق به برسی این ریسک ها خواهیم پرداخت.

\risktbl{اشتباه وارد کردن ورود و خروج}{<30\%}{قابل قبول}

خطای انسانی بسیار رایج بوده و هیچ زمان نمیتوان آن را به صفر رساند. در موارد زیادی ممکن است کاربر و مدیران داده های اشتباهی به سیستم ارائه دهند.
این مشکل باید از طریق طراحی درست سیستم حل بشود.

\risktbl{فراموشی ثبت}{<50\%}{قابل قبول}

خیلی از موارد کاربر فراموش خواهد کرد از دستگاه کنترل تردد استفاده کند. در این حالت به واسطه طراحی مناسب سیستم می‌توان به مدیر درخواست داد یا تیکت زد.
این ریسک نیز قابل قبول بوده و به راحتی قابل حل است.

\risktbl{چیدن اشتباه برد های اسکرام توسط مدیر محصول}{>10\%}{قابل قبول}

اشتباه در چیدن برد اسکرام شرکت را متحمل ضرر زیادی نخواهد کرد. احتمالا با چند بار review ساده کار قابل مشاهده باشد. اما باز هم تیم فنی می‌تواند سیستم را طوری طراحی کند 
که کاربران از ساختاری استاندارد پیروی کنند.

\risktbl{عدم وارد کردن زمان سپری شده توسط نیرو های فنی}{<40\%}{حداقل بودن ریسک}

ممکن است نیرو ها سهوا یا عمدا برای تسک های خود زمان سپری شده را ثبت نکنند یا اشتباه وارد کنند. در این حالت سیستم باید طوری طراحی شود که این 
اتفاق کمتر رخ بدهد یا رخ ندهد.

\risktbl{ایجاد خرابی عمدی توسط یک کاربر با دسترسی خاص}{>5\%}{غیر قابل تحمل}

این مشکل تهدید بزرگی است. برای رفع آن تیم فنی باید به درستی سطوح دسترسی را تنظیم کند.
