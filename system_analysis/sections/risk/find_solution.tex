\subsection{براورد و کاهش هزینه حاصل از ریسک ها}
در این مرحله باید ریسک هایی که شناسایی کردیم و تحلیل کردیم را برطرف کنیم و برای آنها راهکار ارائه دهیم.
این راهکار ها باید عملیاتی و نرم افزاری باشند.

\subsubsection{آسیب پذیری سخت افزار}
رفع مشکلات سخت افزاری و چاره اندیشی برای آنها هزینه مشکلات زیادی به همراه دارد و هزینه بر تر از کار های دیگر است. بهرحال تدابیری را درنظر خواهیم گرفت.

\bt{خرابی سرور پروداکشن:}
برای جلوگیری از خرابی سرور های پروداکشن باید مرتبا سیستم ها مانیتور شوند. نیازداریم با ابزار هایی نظیر Grafana و Prometheus برسی کنیم که در هر لحظه
آیا سرور ها عملکرد مناسبی دارند یا خیر ؟
در صورتی که مشکلی مشاهده شد، سریعا باید برطرف شود تا جدی تر نشود.

\bt{رفتن برق دیتاسنتر:}

برای این مشکل می‌توان از UPS استفاده کرد و تا مدتی برق را برای دیتاسنتر تامین کرد. اینطور تا حدی احتمال و خطر رفتن برق دیتاسنتر ها کمتر می‌شود.

\bt{سوختن حافظه سرور دیتابیس:}

در صورتی که Storage سرور ها به مشکل بخورد ما داده های زیادی را از دست خواهیم داد. برای اینکه داده از دست ندهیم، می‌توانیم از فایل سیستم هایی نظیر RAID و این تیپ
مدل ها استفاده کنیم و داده ها را در Storage های مختلف قرار بدهیم. اگر یکی از آنها سوخت، حتما Backup از آن داشته باشیم.

\bt{اختلال در شبکه دیتاسنتر:}

برای این مورد شبکه دیتاسنتر باید مرتبا مانیتور شود و حدالامکان از تغییرات بر روی آن اجتناب شود. اگر هم بنا بر تغییر است، به شدت تست انجام گیرد و 
مطمعن شویم مشکلی در شبکه ایجاد نمیکند.

\subsubsection{آسیب پذیری نرم افزار}
جلوگیری از ریسک های ناشی از مشکلات نرم افزاری عموما به طراحی تیم توسعه و تدابیر امنیتی تیم امنیت باز خواهد گشت.

\bt{وجود انواع باگ های عملکردی در سیستم و ایجاد اختلال:}

برای اینکه مطمعن شویم اپلیکیشن به درستی کار می‌کند، قبل از Release و Deploy کردن پروژه باید تست های مختلفی از سیستم صورت گیرد.
این تست ها را در مراحل جلو تر شرح خواهیم داد. در کنار تست ها، کاربر ها باید به حداقل آپشن هایی که نیاز دارند دسترسی داشته باشند.
نباید به کاربری بیش از نیازش دسترسی دهیم.

\bt{حملات DoS:}

برای رهایی از حملات DoS می‌توان از انواع Firewall های سخت افزاری و نرم افزاری استفاده کرد. یکی از روش هایی که این مشکل را برطرف می‌کند،استفاده 
از API Gateway هایی نظیر Kong می‌باشد.
اینطور که یک IP Whitelist درست می‌کنیم و فقط کاربران مجاز را به آن معرفی می‌کنیم.
پس از این هیچ کس غیر از آن کاربر ها نمیتواند کاری کند و هر بسته ای که ارسال شود به اصطلاح Drop خواهد شد.

\bt{پر شدن ترافیک سرور:}

به دلیل داخلی بودن سیستم CRM و کم بودن تعداد کاربران چنین ریسکی عملا وجود نخواهد داشت.
اما در صورت وجود چنین ریسکی، می‌توان از Load Balancer ها و Proxy Server ها استفاده کرد.
برای مثال NGINX یا HAProxy گزینه مناسبی خواهد بود.
نود های مختلف اپلیکیشن نیز با ابزاری مانند Kubernetes می‌تواند کلاستر و مدیریت شوند.

\bt{ایجاد کانفلیکت در فرایند دیپلوی کردن نرم افزار و خرابی موقت سیستم:}

در صورتی که تیم DevOps وجود داشته باشد و به درستی فرایند ها و pipline ها را بچیند، چنین مشکلی پیش نخواهد آمد.
زیرا فرایند Deployment خودکار بوده و نیازمند پاس شدن تست های مختلف خواهد بود.

\bt{مورد حمله قرار گرفتن سیستم:}

با استفاده از WAF هایی مانند ModSecurity یا NAXI و پیکره بندی آن بر روی اپلیکیشنی مانند NGINX دیگر چنین مشکلی به اندازه قبل وجود نخواهد داشت.
البته باید توجه داشته باشید ابزاری مانند ModSecurity باید به درستی کانفیگ شود. در غیر این صورت عملکرد False Positive خواهد بود گاها.

\subsubsection{آسیب پذیری اپراتور}
برای رفع مشکلاتی که اپراتور ها میتوانند ایجاد کنند، کافیست دیزاین بهتری داشته باشیم.

\bt{اشتباه وارد کردن ورود و خروج}

ورود و خروج هر فرد ابتدا به ساکن از دستگاه کنترل تردد گرفته می‌شود. مدیر ها در صورت صلاح دید می‌توانند این مقادیر را اصلاح کنند.
در صورتی که کارمندی فراموش کند ورود یا خروج بزند، می‌تواند به مدیران سطح بالا تر تیکت بزند و مشکلش رو مطرح کند. در این حالت با تائید مدیر رکورد مناسب ثبت خواهد شد.
برای همین چنین مشکلی در سیستم CRM ما وجود نخواهد داشت.

\bt{فراموشی ثبت}

در صورتی که کاربری فراموش کند ورود یا خروجی بزند، زیر سیستم مانیتورینگ CRM به مدیران سطوح بالاتر اعلام می‌کند که چنین اتفاقی افتاده است.
این اعلان برای خود کاربر هم می‌رود و به همین ترتیب با اطلاع داشتن دو سمت ماجرا، بالاخره داده ای که باید ثبت خواهد شد.

\bt{چیدن اشتباه برد های اسکرام توسط مدیر محصول}

برای ایجاد برد های اسکرام می‌توان Wizard ایجاد کرد تا مدیر محصول بر اساس استاندارد تعریف شده برد ایجاد کند.

\bt{عدم وارد کردن زمان سپری شده توسط نیرو های فنی}

سیستم مانیتورینگ CRM به نیرو های فنی ای که تسک های خود را به درستی زمان بندی نمی‌کنند اعلان می‌دهد و از آنها درخواست می‌کند که زمان سپری شده خود را وارد کنند.
علاوه بر این بدون وارد کردن زمان سپری شده، نباید تسکی Done شود. 

\bt{ایجاد خرابی عمدی توسط یک کاربر با دسترسی خاص}

با چیدن درست سطوح دسترسی برای کاربران، کاربران اجازه ندارند فرا تر از دسترسی خود کاری انجام بدهند. همین برای این موضوع کافی خواهد بود و دیگر با چنین مشکلی در
سطوح بالا تر مواجه نخواهیم بود.
