\subsection{شناسایی ریسک ها}
تهدیدات و ریسک های مختلفی پروژه CRM را تحدید خواهند کرد که باید از قبل برای آنها چاره اندیشی کرده باشیم.
انواع مختلف ریسک ها می‌تواند گریبان گیر پروژه باشد. در این مرحله به برسی و شناسایی ریسک های مختلف خواهیم پرداخت.

انواع خطر هایی که می‌تواند سیستم CRM را تهدید کند، با دسته بندی می‌نویسیم.

\subsubsection{آسیب پذیری سخت افزار}
در هر برهه زمانی ای ممکن است سخت افزاری که اپلیکیشن بر روی آن است دچار مشکل بشود. در لیست زیر مثال هایی از مشکلات سخت افزاری و زیر ساختی قید می‌شود.

\begin{itemize}
	\item خرابی سرور پروداکشن
	\item رفتن برق دیتاسنتر
	\item سوختن حافظه سرور دیتابیس
	\item اختلال در شبکه دیتاسنتر
\end{itemize}

\subsubsection{آسیب پذیری نرم افزار}
در صورتی که سخت افزار سالم باشم، ریسک ها و مشکلاتی وجود دارد که می‌تواند در سمت نرم افزاری محصول رخ بدهد. در لیست زیر به شناسایی این مشکلات می‌پردازیم.

\begin{itemize}
	\item وجود انواع باگ های عملکردی در سیستم و ایجاد اختلال
	\item حملات DoS
	\item پر شدن ترافیک سرور
	\item ایجاد کانفلیکت در فرایند دیپلوی کردن نرم افزار و خرابی موقت سیستم
	\item مورد حمله قرار گرفتن سیستم
\end{itemize}

\subsubsection{آسیب پذیری اپراتور}
گاهی سخت افزار و نرم افزار سالم بوده، اما اپراتور دچار مشکل می‌شود. برای مثال اشکالات احتمالی زیر را خواهیم داشت.

\begin{itemize}
	\item اشتباه وارد کردن ورود و خروج
	\item فراموشی ثبت
	\item چیدن اشتباه برد های اسکرام توسط مدیر محصول
	\item عدم وارد کردن زمان سپری شده توسط نیرو های فنی
	\item ایجاد خرابی عمدی توسط یک کاربر با دسترسی خاص
\end{itemize}
