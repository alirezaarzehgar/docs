\section{مقدمه ای بر پروژه CRM}
با بزرگ تر شدن شرکت و افزایش تسک ها، تیم مدیریت سازمان ما نیاز به سیستمی دارد که بواسطه آن دپارتمان های مختلف سازمان را مدیریت کند.
برای خیلی از کار های مدیریتی همیشه ابزار هایی وجود خواهد داشت. سازمان ما برای مدیریت کار های خود، نیاز به یک CRM شخصی سازی شده دارد.
CRM درواقع سیستمی است که بواسطه آن شرکت خدمات خود را به کارکنان یا حتی مشتری های خود ارائه می‌دهد.
سیستم های زیادی از قبل توسعه داده شده اند که کار تیم مدیریت را تسهیل کنند. اما مشکلاتی وجود دارد که در ادامه مستند به آنها خواهیم پرداخت.
اغلب نیاز های هر سازمان با سازمان دیگری فرق خواهد کرد. این موضوع زمانی که سازمان ها بزرگ تر می‌شوند بیشتر به چشم خواهد آمد.
از همین رو نیاز ها و استاندارد های سازمان مورد نظر شخصی و یکتا تر خواهد شد. در چنین حالتی ما نیاز به سیستم های انعطاف پذیر تری خواهیم داشت.
سیستمی که سازمان ما مدنظر دارد، باید بتواند به تیم مدیریت در اداره دپارتمان های مختلف کمک کند. در این مقدمه به نیاز های کلی دپارتمان های مختلف خواهیم پرداخت.

\subsubsection{نیازمندی های کلی تیم محصول}
تیم محصول نیاز دارد که گزینه های بسیاری برای مدیریت تیم های مختلف نرم افزاری داشته باشد. طبقه بندی کارمند ها، ساخت تیم های مختلف و تنظیم سطوح دسترسی متفاوت برای هر فرد مهم خواهد بود.
مدیر محصول باید بتواند برای مثال به راحتی متدولوژی ای مانند اسکرام را پیاده کند و تسک های هر فرد را دنبال کند.
پایان هر اسپرینت، باید بتواند گزارشی از عملکرد هر فرد تهیه کند و برای مثال به توسعه دهنده ها یا مهندسینی که بهتر عملکرده اند پاداش بدهد.
این سیستم می‌تواند عملکرد های مختلفی را کنترل کند. نیاز تیم محصول این هست که بتوانیم فیچر های مختلفی را تحت پلاگین به سیستم اضافه کنیم.
برای مثال ممکن است نیاز داشته باشیم که به صورت خودکار، افرادی که از حد مشخصی بیشتر تسک انجام داده باشند را مشخص کنیم.
پس داشبرد تیم محصول نیاز به شخصی سازی و قابلیت توسعه بالایی خواهد داشت.

\subsubsection{نیازمندی های کلی تیم حساب داری}
تیم حساب داری نیاز دارد از سیستم منسوخ شده بروکراسی خود فاصله گرفته و به فرایند خودکار ثبت ورود، خروج و محاسبه حقوق کارمند ها روی بیاورد.
نیاز کلی سازمان حذف بروکراسی و بایگانی دقیق و بی نقص اطلاعات بواسطه خودکار سازی فرایند های حسابداری است.
الگوریتم محاسبه حقوق، اضافه کاری یا دیر کرد می‌تواند در هر سازمان متفاوت باشد.
سیستم باید این قابلیت را داشته باشد که بتوان این الگوریتم را به صورت قابل فهم برای نیرو غیر تکنیکال تعریف کرد. مدیران منابع انسانی و حساب داری باید بتوانند قوانین خود را
به راحتی در سیستم تعریف کنند.
این انعطاف پذیری بالا سیستم هست که نیاز ما را از سیستم های دیگر جدا می‌کند.

\subsubsection{نیازمندی های کلی تیم فنی}
تیم فنی علاوه بر برسی فعالیت های خود بر روی برد اسکرام، نیاز دارد در محیطی مستندات فنی خود را قرار بدهد. چنین بستری باید دارای سطوح دسترسی مختلف باشد.
ایجاد فضا های مختلف و نوشتن مستندات فنی مرتبط با پروژه های مختلف یکی از قابلیت های الزامی ای هست که تیم فنی به آن نیاز خواهند داشت.
علاوه بر آن، تیم فنی نیاز دارد یک سری از گزارشات را هم دنبال کند. برای مثال گزارشات تیم SRE یا تیم امنیت باید برای تیم فنی قابل رویت باشد.
نرم افزار های مختلفی می‌تواند به سیستم وصل بشوند و گزارشات مختلفی داخل داشبرد هر نیرو فنی قرار خواهد گرفت.
برای مثال نیرو ها باید بتوانند عملکرد خود را در پلتفرمی مانند GitLab یا Bitbucket مشاهده کنند.
باید بتوانند حتی به برخی ایمیل های سازمانی پاسخ بدهند یا ببینند چند ایشو یا PR باز داخل داشبرد خود دارند.


\subsection{براورد هزینه ها}
همانطور که اشاره کردیم، ابزار های بسیاری از قبل توسعه یافته اند تا کار تیم های مختلف را ساده کنند. اما تمامی آنها رایگان نیستند.
نمونه ابزار هایی که می‌توان استفاده کرد را ذکر می‌کنیم.

نمونه های بسیار دیگری وجود دارد که در این لیست قید نخواهد شد.

\begin{itemize}
	\item تیم حساب داری برای برسی ورود و خروج کارمندان می‌تواند از نرم افزار KaraWeb استفاده کند.
	\item تیم مدیر محصول برای استفاده از متدولوژی اسکرام می‌تواند از نرم افزار پرمیوم جیرا استفاده کند.
	\item تیم توسعه نرم افزار و باقی تیم های فنی برای نگه داری کد ها و مستندات می‌توانند از GitLab استفاده کنند.
\end{itemize}

در مثال هایی که قید کردیم، کاراوب و جیرا هردو نیاز به پرداخت هزینه دارند.
برای مثال فقط نرم افزار جیرا مبلغی حدود 15250 دلار برای اشتراک سالانه نسخه پرمیوم خود اخز می‌کند. چیزی حدود ۷۰ میلیون تومان در تاریخ نوشته شدن این سند.
با محاسبه این هزینه ها و نیازمنان، باید براورد کنیم که چه کاری درست خواهد بود ؟
آیا توسعه یک سیستم شخصی CRM صرفه اقتصادی می‌کند ؟

\subsubsection{برسی هزینه ها و سود پروژه}
همانطور که اشاره کردیم، برای خرید ابزار های مورد نیاز سالیانه حدود ۸۰ میلیون کمپانی باید هزینه کند. از بین این هزینه ها، خیلی از فیچر ها بلا استفاده بوده
و خیلی از خواسته ها براورده نخواهند شد.
کمپانی چند نیرو با مبلق قانون کار دارد و با درنظر گرفتن قانون کار ۱۰ میلیون، اگر ۳ کارمند به طور تمام وقت کار کنند، اگر قبل از ۲ ماه پروژه به پایان برسد شرکت سود کرده است.
هرچند اگر این زمان بیشتر بشود باز هم شرکد در ضرر نیست. چرا که این سیستم محدود به یک سال نخواهد بود.
علاوه بر آن پس از توسعه این سیستم، می‌توانیم آن را به شرکت های دیگر نیز ارائه دهیم و درامد مضاعفی از فروش اشتراک ها یا پلاگین های آن داشته باشیم.
پس حتی تعداد بیشتری می‌توان روی این پروژه کار کند و این پروژه علاوه بر اینکه نیاز های شرکت را برطرف می‌کنید، خود منبع درامد نیز خواهد بود.

بنا بر این برنامه ما توسعه یک سیستم کامل CRM خواهد بود که نیاز های داخلی خود و دیگران را رفع بکند.
برای اینکار نیاز خواهیم داشت که طراحی ماژولار داشته باشیم و سیستم بر اساس نیاز کاربر تنظیم بشود.

\subsubsection{برسی معایب و مزایا}
در نمودار فوق توضیح می‌دهیم که توسعه یک CRM جدید بومی چه برتری نسبت به استفاده از CRM های موجود در دنیا دارد.
با برسی این جدول توجیه می‌شویم که با شرایط اقتصادی و منطقی کشور، توسعه یک CRM بومی به مراتب به صرفه تر و بهتر از خرید نمونه غیر بومی آن است.

\begin{table}[h!]
\begin{center}
	\caption{مقایسه برتری های سیستم CRM بومی نسبت به ابزار های توسعه داده شده}
	\label{tap:custom_crm_is_better}
	\begin{tabular}{|c|c|}
		\hline
		سیستم بومی & معادل غیر بومی \\
		\hline
		سازگاری با نیازها & عدم سازگاری با همه نیاز های داخلی \\
		\hline
		عدم تحریم یا فیلتر & غالبا ایران را تحریم کرده اند \\
		\hline
		هزینه کمتر & هزینه بیشتر \\
		\hline
		فروش مجدد & دارای کپی رایت \\
		\hline
	\end{tabular}
\end{center}
\end{table}

خالی از لطف نیست که اشاره کنیم اکثر این مزایا بخاطر شرایط کشور  سیستم غلط اداره دولت است و درواقع مزیت توسعه یک سیستم جدید نیست. اشکال سیستم دولت است.
پس از برسی مزایای توسعه یک CRM بومی، بد نیست که راجع به معایب آن نیز صحبت کنیم.

\begin{table}[h!]
	\begin{center}
		\caption{مشکلات توسعه یک CRM بومی}
		\label{tap:old_crm_is_better}
		\begin{tabular}{|l|r|}
			\hline
			سیستم بومی & معادل غیر بومی \\
			\hline
			پر از باگ & کاملا تست شده \\
			\hline
			غیرکیفیت پایین تر & کیفیت بسیار بالا تر \\
			\hline
			پشتیبانی ضعیف & پشتیبانی قوی \\
			\hline
			کمبود بودجه برای توسعه & عدم کمبود بودجه برای توسعه \\
			\hline
			امکان شکست و ضرر & سود مشخص در قبال هزینه \\
			\hline
			احتمال نفوذ & قابل اتکا \\
			\hline
		\end{tabular}
	\end{center}
\end{table}

\clearpage
