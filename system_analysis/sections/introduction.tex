\section{مقدمه ای بر پروژه CRM}
با بزرگ تر شدن شرکت و افزایش تسک ها، تیم مدیریت سازمان ما نیاز به سیستمی دارد که بواسطه آن دپارتمان های مختلف سازمان را مدیریت کند.
برای خیلی از کار های مدیریتی همیشه ابزار هایی وجود خواهد داشت. سازمان ما برای مدیریت کار های خود، نیاز به یک CRM شخصی سازی شده دارد.
CRM درواقع سیستمی است که بواسطه آن شرکت خدمات خود را به کارکنان یا حتی مشتری های خود ارائه می‌دهد.
سیستم های زیادی از قبل توسعه داده شده اند که کار تیم مدیریت را تسهیل کنند. اما مشکلاتی وجود دارد که در ادامه مستند به آنها خواهیم پرداخت.
اغلب نیاز های هر سازمان با سازمان دیگری فرق خواهد کرد. این موضوع زمانی که سازمان ها بزرگ تر می‌شوند بیشتر به چشم خواهد آمد.
از همین رو نیاز ها و استاندارد های سازمان مورد نظر شخصی و یکتا تر خواهد شد. در چنین حالتی ما نیاز به سیستم های انعطاف پذیر تری خواهیم داشت.
سیستمی که سازمان ما مدنظر دارد، باید بتواند به تیم مدیریت در اداره دپارتمان های مختلف کمک کند. در این مقدمه به نیاز های کلی دپارتمان های مختلف خواهیم پرداخت.

\subsubsection{نیازمندی های کلی تیم محصول}
تیم محصول نیاز دارد تا با استفاده از متودولوژی اسکرام اسپرینت ایجاد کند و تسک های مربوط به هر اسپرینت را با جزئیات وارد سیستم کند.
ارائه یک برد اسکرام یا کانبان، یا دنبال کردن تسک ها و برسی عملکرد نیرو ها از کلی ترین نیاز های این دپارتمان خواهد بود.

\subsubsection{نیازمندی های کلی تیم حساب داری}
تیم حساب داری نیاز دارد از سیستم منسوخ شده بروکراسی خود فاصله گرفته و به فرایند خودکار ثبت ورود، خروج و محاسبه حقوق کارمند ها روی بیاورد.
نیاز کلی سازمان حذف بروکراسی و بایگانی دقیق و بی نقص اطلاعات بواسطه خودکار سازی فرایند های حسابداری است.

\subsubsection{نیازمندی های کلی تیم فنی}
تیم فنی علاوه بر برسی فعالیت های خود بر روی برد اسکرام، نیاز دارد در محیطی مستندات فنی خود را قرار بدهد. چنین بستری باید دارای سطوح دسترسی مختلف باشد.
ایجاد فضا های مختلف و نوشتن مستندات فنی مرتبط با پروژه های مختلف یکی از قابلیت های الزامی ای هست که تیم فنی به آن نیاز خواهند داشت.
علاوه بر آن، تیم فنی نیاز دارد یک سری از گزارشات را هم دنبال کند. برای مثال گزارشات تیم SRE یا تیم امنیت باید برای تیم فنی قابل رویت باشد.

\subsection{براورد هزینه ها}
همانطور که اشاره کردیم، ابزار های بسیاری از قبل توسعه یافته اند تا کار تیم های مختلف را ساده کنند. اما تمامی آنها رایگان نیستند.
نمونه ابزار هایی که می‌توان استفاده کرد را ذکر می‌کنیم.

\begin{itemize}
	\item تیم حساب داری برای برسی ورود و خروج کارمندان می‌تواند از نرم افزار KaraWeb استفاده کند.
	\item تیم مدیر محصول برای استفاده از متودولوژی اسکرام می‌تواند از نرم افزار پرمیوم جیرا استفاده کند.
	\item تیم توسعه نرم افزار و باقی تیم های فنی برای نگه داری کد ها و مستندات می‌توانند از GitLab استفاده کنند.
\end{itemize}

در مثال هایی که قید کردیم، کاراوب و جیرا هردو نیاز به پرداخت هزینه دارند.
برای مثال فقط نرم افزار جیرا مبلغی حدود 15250 دلار برای اشتراک سالانه نسخه پرمیوم خود اخز می‌کند. چیزی حدود ۷۰ میلیون تومان در تاریخ نوشته شدن این سند.
با محاسبه این هزینه ها و نیازمنان، باید براورد کنیم که چه کاری درست خواهد بود ؟
آیا توسعه یک سیستم شخصی CRM صرفه اقتصادی می‌کند ؟

\subsubsection{برسی هزینه ها و سود پروژه}
همانطور که اشاره کردیم، برای خرید ابزار های مورد نیاز سالیانه حدود ۸۰ میلیون کمپانی باید هزینه کند. از بین این هزینه ها، خیلی از فیچر ها بلا استفاده بوده
و خیلی از خواسته ها براورده نخواهند شد.
کمپانی چند نیرو با مبلق قانون کار دارد و با درنظر گرفتن قانون کار ۱۰ میلیون، اگر ۳ کارمند به طور تمام وقت کار کنند، اگر قبل از ۲ ماه پروژه به پایان برسد شرکت سود کرده است.
هرچند اگر این زمان بیشتر بشود باز هم شرکد در ضرر نیست. چرا که این سیستم محدود به یک سال نخواهد بود.
علاوه بر آن پس از توسعه این سیستم، می‌توانیم آن را به شرکت های دیگر نیز ارائه دهیم و درامد مضاعفی از فروش اشتراک ها یا پلاگین های آن داشته باشیم.
پس حتی تعداد بیشتری می‌توان روی این پروژه کار کند و این پروژه علاوه بر اینکه نیاز های شرکت را برطرف می‌کنید، خود منبع درامد نیز خواهد بود.

بنا بر این برنامه ما توسعه یک سیستم کامل CRM خواهد بود که نیاز های داخلی خود و دیگران را رفع بکند.
برای اینکار نیاز خواهیم داشت که طراحی ماژولار داشته باشیم و سیستم بر اساس نیاز کاربر تنظیم بشود.
