\section{معرفی UTF8 و Unicode}

\subsection{ASCII}

کامپیوتر ها اعداد رو از هر چیزی بیشتر دوست دارند. تک تک کاراکتر های نوشته شده داخل این نوشته، از نظر کامپیوتر یک سری عدد هستن که از قبل مشخض شده. اما کجا و کی و چطوری ؟
سیستم عامل و کتابخونه های استاندارد در اختیار ما یک سری جدوم در اختیار خودشون دارن که اون تو کاراکتر هایی که ما استفاده میکنیم رو به یک عدد خاص داده.
یکی از ساده ترین این کد گذاری ها ASCII هست. یک جدول ساده داره که داخلش هر عدد به یک کاراکتر از حروف لاتین مرتبطه.
مشکل ASCII این بود که ۸ بیت بیشتر نداشت که از ۷ بیتش استفاده میکرد! درواقع یه چیزی تو مایع های 128 کاراکتر که شامل اعداد و حروف کوچیک و بزرگ انگلیسی بودن.
یک سری کاراکتر چاپ نشدنی هم لا بلای این کد ها وجود داشت که نمیبینمشون اما یک کارای خاص میکنن برامون. مثلا همین اسپیس یا یک خط جدید. از اون هم جادویی تر کاراکتری هست ه کرسر رو عقب و جلو می‌بره.
ولی نهایتا این اصلا کافی نیست. تو ASCII نه خبری از زبون های دیگه بود نه از ایموجی و نه از هیچ چیز دیگه ای...

\subsection{Unicode}

وقتی اینترنت رشد کرد، آدما دوست داشتن با زبان های بومی خودشون از کامپیوتر ها استفاده کنن. اما مشکل اینجا بود که همون طور که آدما زبون خاص خودشون رو داشتن، کامپیوتر ها هم تو شبکه کد گذاری خودشون رو داشتن.
این یکم کار رو سخت میکرد! یعنی مثلا وقتی یک ایمیل برای دوستمون ارسال میکردیم، اگر نمیدونست با چه کدگذاری ای باید اون رو بخونه، یک سری کاراکتر نامفهوم به دستش میرسید.
برای همین موضوع تو شبکه سعی میشد نوع کد گذاری معرفی و مشخص بشه.

طرفای سال ۱۹۸۸ یک کد گذاری جدید اومد به اسم unicode یا کد گذاری جهانی. بر خلاف ASCII که کلا ۸ بیت یا یک بایت برای هر حرف اختصاص میداد، یونیکد حجم خیلی بزرگ تری رو به کاراکتر هاش تخصیص میده!
درواقع یونیکد یک استاندارد هست که اونقدر جا داره که تمام کاراکتر های مورد نیاز رو داخل خودش جا بده. دیگه نیازی نیست هر کسی از انکدینگ خاص خودش استفاده کنه! همه از یک کد گذاری میتونیم استفاده کنیم.

کاراکتر ها در یونیکد تا ۳۲ بایت حافظه میتونن داشته باشن. یک چیزی حدود ۴ میلیون کاراکتر داریم یعنی! ولی چیزی که داخل استاندارد یونیکد مشخص شده ۱۱۱۱۹۹۸ تا کاراکتر بیشتر نیست.

خب این خیلی خوبه. با همون منطق ASCII یک کد گذاری جدید داریم که خیلی کاراکتر های بیشتری رو ساپورت میکنه!

اما یک سوال. ما مجبوریم بخاطر استفاده کردن از یونیکد همیشه چهار برابر حافظه مصرف کنیم ؟

\subsection{UTF-8}

برای مشکل حافظه ای که قبل تر اشاره کردیم، کد گذاری unicode رو درکنار UTF-8 استفاده میکنیم.
درواقع UTF-8 دقیقا کارش همینه که کاراکتر های ۳۲ بیتی unicode رو برامون تو ۸ بیت کد گذاری کنه.
مکانیزم UTF-8 اینطور هست که به محتوا نگاه میکنه و حالا کدگذاریش رو کلاس بندی و دسته بندی میکنه. یه جورایی طول و حجم کاراکتر ها متغییر هستن.

