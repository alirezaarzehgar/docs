%%%%%%%%%%%%%%%%%%%%%%%%%%%%%%%%%%%%%%%%%
% Freeman Curriculum Vitae
% XeLaTeX Template
% Version 3.0 (September 3, 2021)
%
% This template originates from:
% https://www.LaTeXTemplates.com
%
% Authors:
% Vel (vel@LaTeXTemplates.com)
% Alessandro Plasmati
%
% License:
% CC BY-NC-SA 4.0 (https://creativecommons.org/licenses/by-nc-sa/4.0/)
%
%!TEX program = xelatex
% NOTE: this template must be compiled with XeLaTeX rather than PDFLaTeX
% due to the custom fonts used. The line above should ensure this happens
% automatically, but if it doesn't, your LaTeX editor should have a simple toggle
% to switch to using XeLaTeX.
% 
%%%%%%%%%%%%%%%%%%%%%%%%%%%%%%%%%%%%%%%%%

%----------------------------------------------------------------------------------------
%	PACKAGES AND OTHER DOCUMENT CONFIGURATIONS
%----------------------------------------------------------------------------------------

\documentclass[
	10pt, % Default font size, can be between 8pt and 12pt
]{FreemanCV}

\columnratio{0.55, 0.45} % Widths of the two columns, specified here as a ratio summing to 1 to correspond to percentages; adjust as needed for your content 

% Headers and footers can be added with the following commands: \lhead{}, \rhead{}, \lfoot{} and \rfoot{}
% Example right footer:
%\rfoot{\textcolor{headings}{\sffamily Last update: \today. Typeset with Xe\LaTeX}}

%----------------------------------------------------------------------------------------

\begin{document}

\begin{paracol}{2} % Begin two-column mode

%----------------------------------------------------------------------------------------
%	YOUR NAME AND CURRICULUM VITAE TITLE
%----------------------------------------------------------------------------------------

\parbox[][0.11\textheight][c]{\linewidth}{ % Box to hold your name and CV title; change the fixed height as needed to match the colored box to the right
	\centering % Horizontally center text
	
	{\sffamily\Huge Alireza Arzehgar} % Your name
	
	\medskip % Vertical whitespace
	
	{\cursivefont\Huge\textcolor{headings}{Software Developer}}
	
	\vfill % Push content to the top of the box
}

%----------------------------------------------------------------------------------------
%	MAJOR RESEARCH PROJECT
%----------------------------------------------------------------------------------------

\section{Introduction}

{\raggedright\textbf{``I'm a software developer who was UNIX advocated
		and love \LaTeX for creating professional CV and resumes"}\par}

\medskip % Vertical whitespace

I am contributed to lots of open source communities such as FRR, TRex, PTerm and even I'm core contributor of Golang project.
My hobby is reading large and complex technical books like The Linux Programming Interface and Mastering Go. Actually I experienced wide range of 
technologies and development fields like Back-end, Front-end, DevOps, System Administration
Network Administration, Artificial Intellijence, and so on.
I maintain lots of applications and servers for a long time.

\medskip % Extra vertical whitespace before the next section

%----------------------------------------------------------------------------------------
%	WORK EXPERIENCE
%----------------------------------------------------------------------------------------

\section{Work Experience}

% Each job is added with a \jobentry command. Below is an empty one to use as a template:

%\jobentry
%	{} % Duration
%	{} % FT/PT (full time or part time)
%	{} % Employer
%	{} % Job title
%	{} % Description

% All 5 parameters must be supplied but any can be empty if you don't need them

%------------------------------------------------

\jobentry
	{Current, from Jan 2020} % Duration
	{FT} % FT/PT (full time or part time)
	{RedHat Development Team} % Employer
	{System Developer and Cloud Engineer} % Job title
	{As a System Developer I work on kernel modules and bug fixes on RedHat team and expand current features of RedHat Virtualization Enterprise.} % Description

%------------------------------------------------

\jobentry
	{Feb 2018 -- Jan 2020} % Duration
	{FT} % FT/PT (full time or part time)
	{Arvan Cloud} % Employer
	{Cloud Engineer} % Job title
	{This posision involved with maintaining servers and cleating cloud sharing environment to others and offering Software As A Service (SaaS) to clients. I created some custom images on Arvan clouds and maintain big infastructure of them.} % Description

%------------------------------------------------

\jobentry
	{Jul 2014 -- Dec 2018} % Duration
	{PT} % FT/PT (full time or part time)
	{Zharf Pouyan Toos} % Employer
	{Embedded Developer} % Job title
	{In this time I'm working on developing a router and L3 switch based on Advantech devices and ARM RISC switching CPUs} % Description

%----------------------------------------------------------------------------------------
%	REFERENCES
%----------------------------------------------------------------------------------------

\section{References}

%\textit{References available on request} % Uncomment if you'd rather not include references and remove the section below

%------------------------------------------------

% This section is laid out using a table. A \tableentry command adds lines with the following parameters:

%\tableentry{Heading}{Content}{spaceafter}
% All 3 parameters must be supplied but any can be empty if you don't need them
% A "spaceafter" value in the third parameter will add some vertical space -- this is to be used between headings, leave it empty for no extra space

%------------------------------------------------

\begin{supertabular}{r l} % Start a table with two columns, the table will ensure everything is aligned
	
	%------------------------------------------------
	
	\tableentry{}{\textbf{Dr. Isaac Kleiner}}{spaceafter}
	\tableentry{Position}{Professor}{}
	\tableentry{Employer}{\href{https://web.mit.edu/physics/}{Department of Physics}}{}
	\tableentry{}{\href{https://web.mit.edu}{\textit{Massachusetts Institute of Technology}}}{spaceafter}
	\tableentry{Phone}{+1 (617) 253 1000 x5322 (Work)}{}
	\tableentry{Mobile}{+1 (232) 842-3583}{}
	
	%------------------------------------------------
	
	\\ % Additional vertical whitespace between the references
	
	%------------------------------------------------
	
	\tableentry{}{\textbf{Dr. Eli Vance}}{spaceafter}
	\tableentry{Position}{Scientist (HL1)}{}
	\tableentry{Employer}{\href{http://www.bmrf.us}{Black Mesa Research Facility}}{spaceafter}
	\tableentry{Email}{\href{mailto:e.vance@bmrf.us}{e.vance@bmrf.us}}{}
	\tableentry{Phone}{+1 (800) 786-1410 x6235 (Work)}{}
	\tableentry{Mobile}{+1 (201) 632-3901}{}
	
	%------------------------------------------------
	
\end{supertabular}

\medskip % Extra vertical whitespace before the next section

%----------------------------------------------------------------------------------------

\switchcolumn % Switch to the second (right) column

%----------------------------------------------------------------------------------------
%	COLORED CONTACT DETAILS BOX
%----------------------------------------------------------------------------------------

\parbox[top][0.11\textheight][c]{\linewidth}{ % Box to hold the colored box; change the fixed height as needed to match the box to the left
	\colorbox{shade}{ % Create colored box and specify background color
		\begin{supertabular}{@{\hspace{3pt}} p{0.05\linewidth} | p{0.775\linewidth}} % Start a table with two columns, the table will ensure everything is aligned
			\raisebox{-1pt}{\faHome} & Iran, Mashhad, South Motahari 6 \\ % Address
			\raisebox{-1pt}{\faPhone} & +98 915 509 3114 \\ % Phone number
			\raisebox{-1pt}{\small\faEnvelope} & \href{mailto:alirezaarzehgar82@gmail.com}{alirezaarzehgar82@gmail.com} \\ % Email address
			\raisebox{-1pt}{\small\faDesktop} & \href{https://www.LaTeXTemplates.com}{https://www.LaTeXTemplates.com} \\ % Website
			\raisebox{-1pt}{\faGithub} & \href{https://github.com/alirezaarzehgar}{https://github.com/alirezaarzehgar} \\ % GitHub profile
			\raisebox{-1pt}{\faLinkedinSquare} & \href{https://www.linkedin.com/in/alirezaarzehgar}{https://www.linkedin.com/in/alirezaarzehgar} \\ % LinkedIn profile
			% See fontawesome.pdf in the Fonts folder for all icons you can use
		\end{supertabular}
	}
	\vfill % Push content to the top of the box
}

%----------------------------------------------------------------------------------------
%	EDUCATION
%----------------------------------------------------------------------------------------

\section{Education} 

% Each qualification entry is added with a \qualificationentry command. Below is an empty one to use as a template:

%\qualificationentry
%	{} % Duration
%	{} % Degree
%	{} % Honors, achievements or distinctions (e.g. first class honors)
%	{} % Department
%	{} % Institution

% All 5 parameters must be supplied but any can be empty if you don't need them

%------------------------------------------------

\begin{supertabular}{r l} % Start a table with two columns, the table will ensure everything is aligned

	%------------------------------------------------
	
	\qualificationentry
		{1986 -- 1990} % Duration
		{Bachelore} % Degree
		{} % Honors, achievements or distinctions (e.g. first class honors)
		{Computer Science} % Department
		{Islamic Azad University} % Institution
	
	%------------------------------------------------
	
	\qualificationentry
		{1985} % Duration
		{Master of Science} % Degree
		{First Class Honors} % Honors, achievements or distinctions (e.g. first class honors)
		{Theoretical Physics} % Department
		{Yadegaran Emaam} % Institution
	
	%------------------------------------------------


\end{supertabular}

%----------------------------------------------------------------------------------------
%	AWARDS
%----------------------------------------------------------------------------------------

\section{Awards}

% This section is laid out using a table. A \tableentry command adds lines with the following parameters:

%\tableentry{Heading}{Content}{spaceafter}
% All 3 parameters must be supplied but any can be empty if you don't need them
% A "spaceafter" value in the third parameter will add some vertical space -- this is to be used between headings, leave it empty for no extra space

%------------------------------------------------

\begin{supertabular}{r l} % Start a table with two columns, the table will ensure everything is aligned
	
	%------------------------------------------------
	
	\tableentry{1985}{\textbf{Faculty of Science Masters Scholarship}}{}
	\tableentry{}{\textit{Massachusetts Institute of Technology}}{spaceafter}
	
	%------------------------------------------------
	
	\tableentry{1983}{\textbf{Top Achiever Award -- Physics}}{}
	\tableentry{}{\textit{The University of Washington}}{spaceafter}
	
	%------------------------------------------------
	
\end{supertabular}

%----------------------------------------------------------------------------------------
%	COMPUTER SKILLS
%----------------------------------------------------------------------------------------

\section{Software Development Skills} 

% This section is laid out using a table. A \tableentry command adds lines with the following parameters:

%\tableentry{Heading}{Content}{spaceafter}
% All 3 parameters must be supplied but any can be empty if you don't need them
% A "spaceafter" value in the third parameter will add some vertical space -- this is to be used between headings, leave it empty for no extra space

%------------------------------------------------

\begin{supertabular}{r l} % Start a table with two columns, the table will ensure everything is aligned
	
	%------------------------------------------------
	
	\tableentry{Beginner}{\LaTeX, Javascript, PHP, Java}{spaceafter}
	
	%------------------------------------------------
	
	\tableentry{Intermediate}{C, Python, Docker, Networking, }{}
	\tableentry{}{Computer Hardware \& Support}{spaceafter}
	
	%------------------------------------------------
	
	\tableentry{Expert}{Golang, Unix, Virtulization, }{spaceafter}
	
	%------------------------------------------------
	
\end{supertabular}

%----------------------------------------------------------------------------------------
%	COMMUNICATION SKILLS
%----------------------------------------------------------------------------------------

\section{Communication Skills}

% This section is laid out using a table. A \tableentry command adds lines with the following parameters:

%\tableentry{Heading}{Content}{spaceafter}
% All 3 parameters must be supplied but any can be empty if you don't need them
% A "spaceafter" value in the third parameter will add some vertical space -- this is to be used between headings, leave it empty for no extra space

%------------------------------------------------

\begin{supertabular}{r l} % Start a table with two columns, the table will ensure everything is aligned
	
	%------------------------------------------------
	
	\tableentry{Conferences}{Oral Presentation at the Ferdowsi University}{}
	\tableentry{}{LUG Confrence-- 1987}{spaceafter}
	
	%------------------------------------------------
	
	\tableentry{Posters}{Poster at the Meeting of the Mashhad LUG}{}
	\tableentry{}{LUG Society -- 1985}{spaceafter}
	
	%------------------------------------------------
	
\end{supertabular}

%----------------------------------------------------------------------------------------
%	SKILLS DESCRIPTION
%----------------------------------------------------------------------------------------

\section{Skills}

\subsection{Goal Oriented}

I believe in action over long-winded discussions. I listen to everyone's viewpoints and use my judgement to immediately act based on consensus to achieve goals quickly and efficiently.

\subsection{Physical Dexterity}

Manual manipulation of experimental equipment and training within Black Mesa (e.g. the Hazard Course) have contributed to an enjoyment of working with my hands.

\subsection{Passionate}

I have been interested in theoretical physics such as quantum mechanics and relativity from an early age. My education and research have cemented this interest into a passion. I greatly enjoy carrying out fundamental physics research with potential practical applications.


%----------------------------------------------------------------------------------------

\end{paracol} % End two-column mode

%----------------------------------------------------------------------------------------

\end{document}
