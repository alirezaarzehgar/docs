\documentclass{article}

\usepackage{breqn}
\usepackage{amsmath, amsfonts, amsthm}

\newcommand{\dx}{\, \mathrm{d}x}


\usepackage{xepersian}
\settextfont{IRAmir}

\newtheorem{df}{تعریف}[section]

\begin{document}
	سلام سلام بچه ها!
	
	$\phi$
	
	
\begin{align*}
	x = 2 \\
	x = 3 \\
\end{align*}

\[
\sideset{_{.}^{.}}{_{.}^{.}}{\sin}
\]

\[
=\joinrel=
\]

\begin{equation}
\begin{split}
	a & = b \\
	& = c \\
	& = d
\end{split}
\end{equation}

\begin{dgroup}
\begin{dmath}
	\sin^2 x + \cos^2 x = 1
\end{dmath}

\begin{dmath}
	\tan x \cot x = 1
\end{dmath}
\end{dgroup}

\[
\left\lVert \int \frac{\mathrm{e}^x \ln x}{\sin^2 x} \dx \right\rVert
\]

\[
\Bigg[
\]

\begin{df}
سلام گلای تو خونه.
\end{df}

\begin{df}
محصل های نمونه
\end{df}

Hello World

\begin{tabular}{ll}
	hi & hello \\
\end{tabular}

\end{document}